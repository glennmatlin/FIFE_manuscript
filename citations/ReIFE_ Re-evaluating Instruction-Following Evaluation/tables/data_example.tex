% 

\begin{table*}[ht]
  \centering
  \footnotesize
    \begin{tabular}{p{24.5em}c}
    \toprule
    \multicolumn{2}{l}{\textbf{\llmbarnatural}} \\
    \midrule
    \multicolumn{2}{l}{ Explain: Pathos, Ethos, Logos. } \\
    \midrule
    \multicolumn{1}{l}{\textbf{\lime{Output 1} }} & \multicolumn{1}{l}{\textbf{\red{Output 2} }} \\
   
Ethos is the ethical appeal, which is used to convince an audience of the author's credibility or character. 

Pathos is the emotional appeal, which is used to persuade an audience by appealing to their emotions. 

Logos is the appeal to logic and is used to convince an audience by the use of logic and reason.

& \multicolumn{1}{p{24.5em}}{Pathos is a tool of persuasion that is used to appeal to readers' emotions by arousing positive or negative feelings. 

It can be used in rhetoric, literature, film, and other forms of expression.

While pathos is used to draw an emotional response, the other rhetorical appeals appeal to credibility and logic, respectively.

Ethos is the distinguishing character, sentiment, moral nature, or guiding beliefs of a person, group, or institution. 

It is often evident as facts, numbers, or statistics and is used to convince an audience of an argument. } \\
    \midrule
    \multicolumn{2}{l}{\textbf{\adversarial}} \\
    \midrule
    \multicolumn{2}{l}{ Name an animal that is commonly kept as a pet.} \\
    \midrule
    \multicolumn{1}{l}{\textbf{\lime{Output 1}}} & \multicolumn{1}{l}{\textbf{\red{Output 2} }} \\
   
        Dogs are commonly kept as pets.
    & \multicolumn{1}{p{24.5em}}{Dogs, cats, and fish.} \\
    \midrule
    
    \multicolumn{2}{l}{\textbf{\mtbench}} \\
    \midrule
    \multicolumn{2}{l}{ A is the father of B. B is the father of C. What is the relationship between A and C?} \\
    \midrule
    \multicolumn{1}{l}{\textbf{\red{Output 1} }} & \multicolumn{1}{l}{\textbf{\lime{Output 2} }} \\
        A is the grandfather of C.
    & \multicolumn{1}{p{24.5em}}{

        * A is the father of B
        
        * B is the father of C
        
        * Therefore, A is the grandfather of C
        
        So the relationship between A and C is:
        
        Grandfather - Grandson

} \\

    \midrule
    \multicolumn{2}{l}{\textbf{\instrusum}} \\
    \midrule
\multicolumn{2}{l}{ % Use 1 column across, and left-align the text
    \parbox{13cm}{ % Set a paragraph box of width 12cm
         Here is an article:\\
        It is a book with 277 authors but with one shared experience - becoming a new mother in 2020.
        
        The mothers speak about their isolation, ...
        
        Please summarize the above article based on the given requirement:
    }
} \\
    \midrule
    \multicolumn{1}{l}{\textbf{\red{Output 1} }} & \multicolumn{1}{l}{\textbf{\lime{Output 2} }} \\
   
	    
Born in Lockdown is a book written by 277 new mothers in the UK about their experiences of becoming a mother during the pandemic. The mothers share their feelings of isolation, uncertainty, and pressure on their mental health, but also highlight the silver linings and positives that have come out of this joint experience. The project was set up by Bristol novelist Emylia Hall to encourage writing for pleasure and purpose. The book is an unflinchingly honest and moving account of new motherhood during the pandemic, where resilience, love, and hope shine through. The book is available for free download from the Mothership Writers website, with voluntary donations going to the stillbirth and neonatal death charity Sands.
& \multicolumn{1}{p{24.5em}}{Mothers in 2020 have experienced a unique set of challenges due to the pandemic, including isolation, uncertainty, and pressure on their mental health. Despite the hardships, many have found silver linings and positives that have come out of the shared experience. Through the writing project Born in Lockdown, 277 mothers from across the UK have shared their stories, expressing a range of emotions including resilience, love, and hope. } \\
    \bottomrule
    \end{tabular}%
  \caption{Selected examples from our datasets (\S\ref{sec:data_models}). The green label indicates the output is the preferred output.}
  \label{tab:dataset_ex_1}
\end{table*}%
